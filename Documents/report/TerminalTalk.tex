\documentclass{article}







%%%%%%%%%%%%%%%%%%%%%%%%%%%%%%%%%%%%%%%%%%%%%%%%%%%%%%%%%%%%%%%%%%%%%%%%%%%%%%%%
%
%
%							Include	Packages
%
%
%%%%%%%%%%%%%%%%%%%%%%%%%%%%%%%%%%%%%%%%%%%%%%%%%%%%%%%%%%%%%%%%%%%%%%%%%%%%%%%%

% Packages for API
\usepackage{fixltx2e}
\usepackage{calc}
\usepackage{doxygen}
\usepackage[export]{adjustbox} % also loads graphicx
\usepackage{graphicx}
\usepackage[utf8]{inputenc}
\usepackage{makeidx}
\usepackage{multicol}
\usepackage{multirow}
\PassOptionsToPackage{warn}{textcomp}
\usepackage{textcomp}
\usepackage[nointegrals]{wasysym}
\usepackage[table]{xcolor}
\usepackage{pgf,tikz}
\usetikzlibrary{arrows}
\usepackage{csquotes}

%%%%% Package for displaying code snippets
\usepackage{listings}
% And for syntax highlighting of code...
\usepackage{color}
\definecolor{mygreen}{rgb}{0,0.6,0}
\definecolor{mygray}{rgb}{0.5,0.5,0.5}
\definecolor{mymauve}{rgb}{0.58,0,0.82}
% Format/ Style definitions for Code snippets
\lstset{ %
	backgroundcolor=\color{white},   % choose the background color; you must add \usepackage{color} or \usepackage{xcolor}
	basicstyle=\footnotesize,        % the size of the fonts that are used for the code
	breakatwhitespace=false,         % sets if automatic breaks should only happen at whitespace
	breaklines=true,                 % sets automatic line breaking
	captionpos=b,                    % sets the caption-position to bottom
	commentstyle=\color{mygreen},    % comment style
	deletekeywords={...},            % if you want to delete keywords from the given language
	escapeinside={\%*}{*)},          % if you want to add LaTeX within your code
	extendedchars=true,              % lets you use non-ASCII characters; for 8-bits encodings only, does not work with UTF-8
	frame=single,	                 % adds a frame around the code
	keepspaces=true,                 % keeps spaces in text, useful for keeping indentation of code (possibly needs columns=flexible)
	keywordstyle=\color{blue},       % keyword style
	language=Octave,                 % the language of the code
	otherkeywords={*,...},           % if you want to add more keywords to the set
	numbers=left,                    % where to put the line-numbers; possible values are (none, left, right)
	numbersep=5pt,                   % how far the line-numbers are from the code
	numberstyle=\tiny\color{mygray}, % the style that is used for the line-numbers
	rulecolor=\color{black},         % if not set, the frame-color may be changed on line-breaks within not-black text (e.g. comments (green here))
	showspaces=false,                % show spaces everywhere adding particular underscores; it overrides 'showstringspaces'
	showstringspaces=false,          % underline spaces within strings only
	showtabs=false,                  % show tabs within strings adding particular underscores
	stepnumber=2,                    % the step between two line-numbers. If it's 1, each line will be numbered
	stringstyle=\color{mymauve},     % string literal style
	tabsize=2,	                   % sets default tabsize to 2 spaces
	title=\lstname                   % show the filename of files included with \lstinputlisting; also try caption instead of title
}


% Font selection
\usepackage[T1]{fontenc}
\usepackage[scaled=.90]{helvet}
\usepackage{courier}
\usepackage{amssymb,amsmath,amsfonts,amsthm}
\usepackage{sectsty}







%%%%%%%%%%%%%%%%%%%%%%%%%%%%%%%%%%%%%%%%%%%%%%%%%%%%%%%%%%%%%%%%%%%%%%%%%%%%%%%%
%
%
%							Document Style and Definitions
%
%
%%%%%%%%%%%%%%%%%%%%%%%%%%%%%%%%%%%%%%%%%%%%%%%%%%%%%%%%%%%%%%%%%%%%%%%%%%%%%%%%


%%%%%%%%%%%%%%%%%%%%%%%%%%%%%%%
%	Define Title
%%%%%%%%%%%%%%%%%%%%%%%%%%%%%%%
\renewcommand{\familydefault}{\rmdefault}
\allsectionsfont{%
  \fontseries{bc}\selectfont%
  \color{darkgray}%
}
\renewcommand{\DoxyLabelFont}{%
  \fontseries{bc}\selectfont%
  \color{darkgray}%
}
\newcommand{\+}{\discretionary{\mbox{\scriptsize$\hookleftarrow$}}{}{}}

% Page & text layout
\usepackage{geometry}
\geometry{%
  a4paper,%
  top=2.5cm,%
  bottom=2.5cm,%
  left=2.5cm,%
  right=2.5cm%
}
\tolerance=750
\hfuzz=15pt
\hbadness=750
\setlength{\emergencystretch}{15pt}
\setlength{\parindent}{0cm}
\setlength{\parskip}{3ex plus 2ex minus 2ex}
\makeatletter
\renewcommand{\paragraph}{%
  \@startsection{paragraph}{4}{0ex}{-1.0ex}{1.0ex}{%
    \normalfont\normalsize\bfseries\SS@parafont%
  }%
}
\renewcommand{\subparagraph}{%
  \@startsection{subparagraph}{5}{0ex}{-1.0ex}{1.0ex}{%
    \normalfont\normalsize\bfseries\SS@subparafont%
  }%
}
\makeatother

% Headers & footers
\usepackage{fancyhdr}
\pagestyle{fancyplain}
\fancyhead[LE]{\fancyplain{}{\bfseries\thepage}}
\fancyhead[CE]{\fancyplain{}{}}
\fancyhead[RE]{\fancyplain{}{\bfseries\leftmark}}
\fancyhead[LO]{\fancyplain{}{\bfseries\rightmark}}
\fancyhead[CO]{\fancyplain{}{}}
\fancyhead[RO]{\fancyplain{}{\bfseries\thepage}}
\fancyfoot[LE]{\fancyplain{}{}}
\fancyfoot[CE]{\fancyplain{}{}}
\fancyfoot[RE]{\fancyplain{}{\bfseries\scriptsize Generated by Doxygen }}
\fancyfoot[LO]{\fancyplain{}{\bfseries\scriptsize Terminal Talk }}
\fancyfoot[CO]{\fancyplain{}{}}
\fancyfoot[RO]{\fancyplain{}{}}
\renewcommand{\footrulewidth}{0.4pt}

\renewcommand{\sectionmark}[1]{%
  \markright{\thesection\ #1}%
}

% Indices & bibliography
\usepackage{natbib}
\usepackage[titles]{tocloft}
\setcounter{tocdepth}{3}
\setcounter{secnumdepth}{5}
\makeindex



%%%%%%%%%%%%%%%%%%%%%%%%%%%%%%%
%	Define Title
%%%%%%%%%%%%%%%%%%%%%%%%%%%%%%%
\newcommand{\horrule}[1]{\rule{\linewidth}{#1}} % Create horizontal rule command with 1 argument of height

\title{	
	\normalfont \normalsize 
	\textsc{University of Mississippi, Department of Electrical Engineering} \\ [25pt] % Your university, school and/or department name(s)
	\horrule{0.5pt} \\[0.4cm] % Thin top horizontal rule
	\huge El E 425 --- TerminalTalk  \\ % The assignment title
	\horrule{2pt} \\[0.5cm] % Thick bottom horizontal rule
}

\author{Bryan Harper} % Your name

\date{\normalsize\today} % Today's date or a custom date




% Hyperlinks (required, but should be loaded last)
\usepackage{ifpdf}
\ifpdf
  \usepackage[pdftex,pagebackref=true]{hyperref}
\else
  \usepackage[ps2pdf,pagebackref=true]{hyperref}
\fi
\hypersetup{%
  colorlinks=true,%
  linkcolor=blue,%
  citecolor=blue,%
  unicode%
}

% Custom commands
\newcommand{\clearemptydoublepage}{%
  \newpage{\pagestyle{empty}\cleardoublepage}%
}

\usepackage{caption}
\captionsetup{labelsep=space,justification=centering,font={bf},singlelinecheck=off,skip=4pt,position=top}

%===== C O N T E N T S =====

\begin{document}

% Titlepage & ToC
\hypersetup{pageanchor=false,
             bookmarksnumbered=true,
             pdfencoding=unicode
            }
\pagenumbering{roman}


\maketitle


\clearemptydoublepage
\tableofcontents
\clearemptydoublepage
\pagenumbering{arabic}
\hypersetup{pageanchor=true}

%--- Begin generated contents ---
\part{}
\section{Introduction}
\subsection{Overview}
Test 123
\subsection{The End Product}
In order to determine the application architecture and protocol with which to implement the software it is first necessary to develop a clear picture of what is being built. The following is a description of what an end user can expect when running the \textsl{TerminalTalk} application.

The application will be completely run from the command line (terminal on Linux). There will be no additional graphical user interface beyond what the terminal provides (ANSI characters, cursor animation, colors, etc.). Thus, any interaction the user has with the application will be through their keyboard, in the form of normal input or specialized commands. 

Upon starting the application --- with the option of specifying a moniker (username) --- the user will be connected to a \emph{chat room}. The user's arrival will be announced to all users currently connected:

\begin{displayquote}
 \texttt{[moniker] has entered.}
\end{displayquote}

There will also be available to the user a non-intrusive prompt for the user to deliver a message to the room:

\begin{displayquote}
 \texttt{Talk:}
\end{displayquote}
 
 If the user types a message and then presses \texttt{`Enter'} on their keyboard, the user will simply see their message displayed after the prompt, and then another prompt on a new line. 
 
\begin{displayquote}
 \texttt{Talk: [message inputed by user]} \\
 \texttt{Talk:}
\end{displayquote}

However, any other user in the room will see something like:

\begin{displayquote}
	\texttt{\textbf{\textcolor{cyan}{[moniker1]:}} [message inputed by user1]} \\
	\texttt{Talk:}
\end{displayquote}

Note that the first user's moniker is colored. When a user first begins the application the program randomly assigns a color to the user. However, if a user wishes to change their assigned color, they can use the command \texttt{\textbackslash setcolor}:

\begin{displayquote}
	\texttt{Talk: \textbackslash  setcolor green} \\
	\texttt{\emph{Your color has been set to \textcolor{green}{green}.}}
\end{displayquote}

There is also a command to change the user's moniker:

\begin{displayquote}
	\texttt{Talk: \textbackslash  setmoniker Alan} \\
	\texttt{\emph{Hello Alan.}}
\end{displayquote}

And a command to disconnect:

\begin{displayquote}
	\texttt{Talk: \textbackslash  disconnect} \\
	\texttt{\emph{You have disconnected.}}
\end{displayquote}

Seen from the perspective of other connected users, the output resulting from a user running the above commands might look like this:

\begin{displayquote}
	\texttt{\emph{Alan has entered.}} \\
	\texttt{\textcolor{cyan}{Anonymous:} Hello everyone!} \\
	\texttt{\textcolor{green}{Anonymous:} How's it going?} \\
	\texttt{\textcolor{green}{Alan:} Uh oh, I've got to run!} \\
	\texttt{\emph{Alan has exited.}}
\end{displayquote}

It's expected that the chat room will always be available for a user to connect with at any time. It is also expected that messages are transferred without error. To see the necessity of this, consider if the letter `e` were to be dropped from the word `appeal' in a message. Clearly, such an error could lead to a detrimental misunderstanding.

In a similar vein, it is necessary that messages are received in order. If, for example, a disconnect command were received out of order and before another message, the user would be disconnected before their message got through. This is clearly an unacceptable result. 



\section{Theoretical Discussion}
\subsection{Architecture}
\subsection{Application Protocol}

\section{Implementation}

\part{API}

\section{Namespace Index}
\subsection{Packages}
Here are the packages with brief descriptions (if available)\+:\begin{DoxyCompactList}
\item\contentsline{section}{\hyperlink{namespaceclient}{client} \\*Functions and definitions specific to Terminal\+Talk client }{\pageref{namespaceclient}}{}
\item\contentsline{section}{\hyperlink{namespacefont}{font} \\*A collection of functions that style fonts in terminal }{\pageref{namespacefont}}{}
\item\contentsline{section}{\hyperlink{namespaceserver}{server} \\*Functions and definitions specific to Terminal\+Talk server }{\pageref{namespaceserver}}{}
\end{DoxyCompactList}

\section{Class Index}
\section{Class List}
Here are the classes, structs, unions and interfaces with brief descriptions\+:\begin{DoxyCompactList}
\item\contentsline{section}{\hyperlink{classfont_1_1constants_1_1Colors}{font.\+constants.\+Colors} \\*Contains constants for A\+N\+SI color codes }{\pageref{classfont_1_1constants_1_1Colors}}{}
\item\contentsline{section}{\hyperlink{classserver_1_1Orator_1_1Orator}{server.\+Orator.\+Orator} \\*Stores information about a connected client }{\pageref{classserver_1_1Orator_1_1Orator}}{}
\item\contentsline{section}{\hyperlink{classfont_1_1constants_1_1Styles}{font.\+constants.\+Styles} \\*Contains constants for A\+N\+SI text style codes }{\pageref{classfont_1_1constants_1_1Styles}}{}
\end{DoxyCompactList}

\section{File Index}
\section{File List}
Here is a list of all documented files with brief descriptions\+:\begin{DoxyCompactList}
\item\contentsline{section}{/home/bryan/\+Terminal\+Talk/font/\hyperlink{constants_8py}{constants.\+py} \\*Defines classes contains A\+N\+SI constants for formating text }{\pageref{constants_8py}}{}
\item\contentsline{section}{/home/bryan/\+Terminal\+Talk/font/\hyperlink{functions_8py}{functions.\+py} \\*Defines functions for surronding strings with A\+N\+SI codes for formating text }{\pageref{functions_8py}}{}
\end{DoxyCompactList}

\section{Namespace Documentation}
\hypertarget{namespaceclient}{}\subsection{client Namespace Reference}
\label{namespaceclient}\index{client@{client}}


Functions and definitions specific to Terminal\+Talk client.  




\subsubsection{Detailed Description}
Functions and definitions specific to Terminal\+Talk client. 
\hypertarget{namespacefont}{}\subsection{font Namespace Reference}
\label{namespacefont}\index{font@{font}}


A collection of functions that style fonts in terminal.  




\subsubsection{Detailed Description}
A collection of functions that style fonts in terminal. 

Each function flanks the inputed text with A\+N\+SI codes. 
\hypertarget{namespaceserver}{}\subsection{server Namespace Reference}
\label{namespaceserver}\index{server@{server}}


Functions and definitions specific to Terminal\+Talk server.  




\subsubsection{Detailed Description}
Functions and definitions specific to Terminal\+Talk server. 
\section{Class Documentation}
\hypertarget{classfont_1_1constants_1_1Colors}{}\subsection{font.\+constants.\+Colors Class Reference}
\label{classfont_1_1constants_1_1Colors}\index{font.\+constants.\+Colors@{font.\+constants.\+Colors}}


Contains constants for A\+N\+SI color codes.  


\subsubsection*{Static Public Attributes}
\begin{DoxyCompactItemize}
\item 
string {\bfseries B\+L\+A\+CK} = \textquotesingle{}\textbackslash{}033\mbox{[}30m\mbox{]}\textquotesingle{}\hypertarget{classfont_1_1constants_1_1Colors_afb0947d4badc4d61510141f299ae342f}{}\label{classfont_1_1constants_1_1Colors_afb0947d4badc4d61510141f299ae342f}

\item 
string {\bfseries R\+ED} = \textquotesingle{}\textbackslash{}033\mbox{[}31m\textquotesingle{}\hypertarget{classfont_1_1constants_1_1Colors_ab71c856d5c4608461c28c583dd7801b6}{}\label{classfont_1_1constants_1_1Colors_ab71c856d5c4608461c28c583dd7801b6}

\item 
string {\bfseries G\+R\+E\+EN} = \textquotesingle{}\textbackslash{}033\mbox{[}32m\textquotesingle{}\hypertarget{classfont_1_1constants_1_1Colors_a36b2a63f5603bddc4b00def85db005b6}{}\label{classfont_1_1constants_1_1Colors_a36b2a63f5603bddc4b00def85db005b6}

\item 
string {\bfseries Y\+E\+L\+L\+OW} = \textquotesingle{}\textbackslash{}033\mbox{[}33m\textquotesingle{}\hypertarget{classfont_1_1constants_1_1Colors_a7325d06cee4dc3a6097bf584d7a0c086}{}\label{classfont_1_1constants_1_1Colors_a7325d06cee4dc3a6097bf584d7a0c086}

\item 
string {\bfseries B\+L\+UE} = \textquotesingle{}\textbackslash{}033\mbox{[}34m\textquotesingle{}\hypertarget{classfont_1_1constants_1_1Colors_a46ce2e0fc3e5c40075cae590eaf85293}{}\label{classfont_1_1constants_1_1Colors_a46ce2e0fc3e5c40075cae590eaf85293}

\item 
string {\bfseries M\+A\+G\+E\+N\+TA} = \textquotesingle{}\textbackslash{}033\mbox{[}35m\textquotesingle{}\hypertarget{classfont_1_1constants_1_1Colors_a8fa42a6991d5157672c51a9b953279c6}{}\label{classfont_1_1constants_1_1Colors_a8fa42a6991d5157672c51a9b953279c6}

\item 
string {\bfseries C\+Y\+AN} = \textquotesingle{}\textbackslash{}033\mbox{[}36m\textquotesingle{}\hypertarget{classfont_1_1constants_1_1Colors_a3d199ec328ec474f2aa6919aed63c9be}{}\label{classfont_1_1constants_1_1Colors_a3d199ec328ec474f2aa6919aed63c9be}

\item 
string {\bfseries W\+H\+I\+TE} = \textquotesingle{}\textbackslash{}033\mbox{[}37m\textquotesingle{}\hypertarget{classfont_1_1constants_1_1Colors_a10f7fbd2ec87e066fe3fba002e2191b6}{}\label{classfont_1_1constants_1_1Colors_a10f7fbd2ec87e066fe3fba002e2191b6}

\end{DoxyCompactItemize}


\subsection{Detailed Description}
Contains constants for A\+N\+SI color codes. 

Prefix a string with color constant to change subsequent text to the color. Example\+: print( font.\+Colors.\+R\+ED + \char`\"{}\+This text will be red.\char`\"{}) 

The documentation for this class was generated from the following file\+:\begin{DoxyCompactItemize}
\item 
/\+Terminal\+Talk/font/\hyperlink{constants_8py}{constants.\+py}\end{DoxyCompactItemize}

\hypertarget{classserver_1_1Orator_1_1Orator}{}\section{server.\+Orator.\+Orator Class Reference}
\label{classserver_1_1Orator_1_1Orator}\index{server.\+Orator.\+Orator@{server.\+Orator.\+Orator}}


Stores information about a connected client.  


\subsection*{Public Member Functions}
\begin{DoxyCompactItemize}
\item 
def \hyperlink{classserver_1_1Orator_1_1Orator_a173c8497b251bba46514f28e089597eb}{\+\_\+\+\_\+init\+\_\+\+\_\+} (self, telegraph)
\begin{DoxyCompactList}\small\item\em Constructor for \hyperlink{classserver_1_1Orator_1_1Orator}{Orator} class. \end{DoxyCompactList}\end{DoxyCompactItemize}
\subsection*{Public Attributes}
\begin{DoxyCompactItemize}
\item 
{\bfseries telegraph}\hypertarget{classserver_1_1Orator_1_1Orator_a87ae5421efc0e5f4ccce21d04cb921f4}{}\label{classserver_1_1Orator_1_1Orator_a87ae5421efc0e5f4ccce21d04cb921f4}

\item 
{\bfseries moniker}\hypertarget{classserver_1_1Orator_1_1Orator_a153d7720e2e5f6b7c41c67333a1390ba}{}\label{classserver_1_1Orator_1_1Orator_a153d7720e2e5f6b7c41c67333a1390ba}

\item 
{\bfseries color}\hypertarget{classserver_1_1Orator_1_1Orator_a1a5fc663a1e5aaa7d58fbcf05829d882}{}\label{classserver_1_1Orator_1_1Orator_a1a5fc663a1e5aaa7d58fbcf05829d882}

\end{DoxyCompactItemize}


\subsection{Detailed Description}
Stores information about a connected client. 

An \hyperlink{classserver_1_1Orator_1_1Orator}{Orator} objects is used to represent a client. Each client has a moniker and color associated with them. The moniker is provided by the client. The color is assigned randomly each time the client connects. 

\subsection{Constructor \& Destructor Documentation}
\index{server\+::\+Orator\+::\+Orator@{server\+::\+Orator\+::\+Orator}!\+\_\+\+\_\+init\+\_\+\+\_\+@{\+\_\+\+\_\+init\+\_\+\+\_\+}}
\index{\+\_\+\+\_\+init\+\_\+\+\_\+@{\+\_\+\+\_\+init\+\_\+\+\_\+}!server\+::\+Orator\+::\+Orator@{server\+::\+Orator\+::\+Orator}}
\subsubsection[{\texorpdfstring{\+\_\+\+\_\+init\+\_\+\+\_\+(self, telegraph)}{__init__(self, telegraph)}}]{\setlength{\rightskip}{0pt plus 5cm}def server.\+Orator.\+Orator.\+\_\+\+\_\+init\+\_\+\+\_\+ (
\begin{DoxyParamCaption}
\item[{}]{self, }
\item[{}]{telegraph}
\end{DoxyParamCaption}
)}\hypertarget{classserver_1_1Orator_1_1Orator_a173c8497b251bba46514f28e089597eb}{}\label{classserver_1_1Orator_1_1Orator_a173c8497b251bba46514f28e089597eb}


Constructor for \hyperlink{classserver_1_1Orator_1_1Orator}{Orator} class. 


\begin{DoxyParams}{Parameters}
{\em telegraph} & The socket used to connect with client. \\
\hline
\end{DoxyParams}


The documentation for this class was generated from the following file\+:\begin{DoxyCompactItemize}
\item 
/home/bryan/\+Terminal\+Talk/server/\hyperlink{Orator_8py}{Orator.\+py}\end{DoxyCompactItemize}

\hypertarget{classfont_1_1constants_1_1Styles}{}\section{font.\+constants.\+Styles Class Reference}
\label{classfont_1_1constants_1_1Styles}\index{font.\+constants.\+Styles@{font.\+constants.\+Styles}}


Contains constants for A\+N\+SI text style codes.  


\subsection*{Static Public Attributes}
\begin{DoxyCompactItemize}
\item 
string {\bfseries R\+E\+S\+ET} = \textquotesingle{}\textbackslash{}033\mbox{[}0m\textquotesingle{}\hypertarget{classfont_1_1constants_1_1Styles_ad29d2bc2798787e0440842fb4a481fb6}{}\label{classfont_1_1constants_1_1Styles_ad29d2bc2798787e0440842fb4a481fb6}

\item 
string {\bfseries B\+O\+LD} = \textquotesingle{}\textbackslash{}033\mbox{[}1m\textquotesingle{}\hypertarget{classfont_1_1constants_1_1Styles_a0f0afd96809e2254562ac6cc0fe119ab}{}\label{classfont_1_1constants_1_1Styles_a0f0afd96809e2254562ac6cc0fe119ab}

\item 
string {\bfseries B\+O\+L\+D\+\_\+\+O\+FF} = \textquotesingle{}\textbackslash{}033\mbox{[}22m\textquotesingle{}\hypertarget{classfont_1_1constants_1_1Styles_a574cd81c9ce162e8d28e10c8da5f7c88}{}\label{classfont_1_1constants_1_1Styles_a574cd81c9ce162e8d28e10c8da5f7c88}

\item 
string {\bfseries I\+T\+A\+L\+IC} = \textquotesingle{}\textbackslash{}033\mbox{[}3m\textquotesingle{}\hypertarget{classfont_1_1constants_1_1Styles_a055d7fd4d1d90090fe3b628353e1dd26}{}\label{classfont_1_1constants_1_1Styles_a055d7fd4d1d90090fe3b628353e1dd26}

\item 
string {\bfseries I\+T\+A\+L\+I\+C\+\_\+\+O\+FF} = \textquotesingle{}\textbackslash{}033\mbox{[}23m\textquotesingle{}\hypertarget{classfont_1_1constants_1_1Styles_a38293343c5f14e2a1a055b64532f2e3d}{}\label{classfont_1_1constants_1_1Styles_a38293343c5f14e2a1a055b64532f2e3d}

\item 
string {\bfseries U\+N\+D\+E\+R\+L\+I\+NE} = \textquotesingle{}\textbackslash{}033\mbox{[}4m\textquotesingle{}\hypertarget{classfont_1_1constants_1_1Styles_a3c88ee0f4b74663abb8336057ce907b8}{}\label{classfont_1_1constants_1_1Styles_a3c88ee0f4b74663abb8336057ce907b8}

\item 
string {\bfseries U\+N\+D\+E\+R\+L\+I\+N\+E\+\_\+\+O\+FF} = \textquotesingle{}\textbackslash{}033\mbox{[}24m\textquotesingle{}\hypertarget{classfont_1_1constants_1_1Styles_a5fabd3fadc86822d9a401c4bec80a966}{}\label{classfont_1_1constants_1_1Styles_a5fabd3fadc86822d9a401c4bec80a966}

\item 
string {\bfseries S\+T\+R\+I\+K\+E\+T\+H\+R\+O\+U\+GH} = \textquotesingle{}\textbackslash{}033\mbox{[}9m\textquotesingle{}\hypertarget{classfont_1_1constants_1_1Styles_a49438d7dd8b4e9de783717b6d4812025}{}\label{classfont_1_1constants_1_1Styles_a49438d7dd8b4e9de783717b6d4812025}

\item 
string {\bfseries S\+T\+R\+I\+K\+E\+T\+H\+R\+O\+U\+H\+G\+\_\+\+O\+FF} = \textquotesingle{}\textbackslash{}033\mbox{[}29m\textquotesingle{}\hypertarget{classfont_1_1constants_1_1Styles_a5406984ee410e185cd36701be2c7c64c}{}\label{classfont_1_1constants_1_1Styles_a5406984ee410e185cd36701be2c7c64c}

\item 
string {\bfseries I\+N\+V\+E\+R\+SE} = \textquotesingle{}\textbackslash{}033\mbox{[}7m\textquotesingle{}\hypertarget{classfont_1_1constants_1_1Styles_a616d45cb55b02a8503229fca77dad35a}{}\label{classfont_1_1constants_1_1Styles_a616d45cb55b02a8503229fca77dad35a}

\item 
string {\bfseries I\+N\+V\+E\+R\+S\+E\+\_\+\+O\+FF} = \textquotesingle{}\textbackslash{}033\mbox{[}27m\textquotesingle{}\hypertarget{classfont_1_1constants_1_1Styles_ab3b2d4eedd8b2a3a437689de4c4262fc}{}\label{classfont_1_1constants_1_1Styles_ab3b2d4eedd8b2a3a437689de4c4262fc}

\end{DoxyCompactItemize}


\subsection{Detailed Description}
Contains constants for A\+N\+SI text style codes. 

Prefix a string with a style constant to change subsequent text style. Example\+: print( font.\+Styles.\+B\+O\+LD + \char`\"{}\+This text will be bold.\char`\"{}) 

The documentation for this class was generated from the following file\+:\begin{DoxyCompactItemize}
\item 
/home/bryan/\+Terminal\+Talk/font/\hyperlink{constants_8py}{constants.\+py}\end{DoxyCompactItemize}

\section{File Documentation}
\hypertarget{eavesdrop_8py}{}\section{/home/bryan/\+Terminal\+Talk/client/eavesdrop.py File Reference}
\label{eavesdrop_8py}\index{/home/bryan/\+Terminal\+Talk/client/eavesdrop.\+py@{/home/bryan/\+Terminal\+Talk/client/eavesdrop.\+py}}
\subsection*{Functions}
\begin{DoxyCompactItemize}
\item 
def \hyperlink{eavesdrop_8py_a54056409aca4726465412b58f7af1e58}{client.\+eavesdrop.\+eavesdrop} ()\hypertarget{eavesdrop_8py_a54056409aca4726465412b58f7af1e58}{}\label{eavesdrop_8py_a54056409aca4726465412b58f7af1e58}

\begin{DoxyCompactList}\small\item\em Monitors for user input from terminal. \end{DoxyCompactList}\end{DoxyCompactItemize}

\hypertarget{constants_8py}{}\subsection{/\+Terminal\+Talk/font/constants.py File Reference}
\label{constants_8py}\index{/\+Terminal\+Talk/font/constants.\+py@{/\+Terminal\+Talk/font/constants.\+py}}


Defines classes contains A\+N\+SI constants for formating text.  


\subsubsection*{Classes}
\begin{DoxyCompactItemize}
\item 
class \hyperlink{classfont_1_1constants_1_1Styles}{font.\+constants.\+Styles}
\begin{DoxyCompactList}\small\item\em Contains constants for A\+N\+SI text style codes. \end{DoxyCompactList}\item 
class \hyperlink{classfont_1_1constants_1_1Colors}{font.\+constants.\+Colors}
\begin{DoxyCompactList}\small\item\em Contains constants for A\+N\+SI color codes. \end{DoxyCompactList}\end{DoxyCompactItemize}


\subsubsection{Detailed Description}
Defines classes contains A\+N\+SI constants for formating text. 


\hypertarget{functions_8py}{}\section{/home/bryan/\+Terminal\+Talk/font/functions.py File Reference}
\label{functions_8py}\index{/home/bryan/\+Terminal\+Talk/font/functions.\+py@{/home/bryan/\+Terminal\+Talk/font/functions.\+py}}


Defines functions for surronding strings with A\+N\+SI codes for formating text.  


\subsection*{Functions}
\begin{DoxyCompactItemize}
\item 
def \hyperlink{functions_8py_ac4016dd2373b79081f110154a3fe2c96}{font.\+functions.\+blue} (txt)
\begin{DoxyCompactList}\small\item\em Makes inputed string blue. \end{DoxyCompactList}\item 
def \hyperlink{functions_8py_a0434b99c6b721734c21210540c8c7dce}{font.\+functions.\+cyan} (txt)
\begin{DoxyCompactList}\small\item\em Makes inputed string cyan. \end{DoxyCompactList}\item 
def \hyperlink{functions_8py_a1a75ea7211b4c6d154d2377b7913f3d2}{font.\+functions.\+green} (txt)
\begin{DoxyCompactList}\small\item\em Makes inputed string green. \end{DoxyCompactList}\item 
def \hyperlink{functions_8py_abb25d9db89cedb6b2fb0b8371ae196b1}{font.\+functions.\+magenta} (txt)
\begin{DoxyCompactList}\small\item\em Makes inputed string magenta. \end{DoxyCompactList}\item 
def \hyperlink{functions_8py_a92cba3e5f82247d09e1e045ce141ec60}{font.\+functions.\+red} (txt)
\begin{DoxyCompactList}\small\item\em Makes inputed string red. \end{DoxyCompactList}\item 
def \hyperlink{functions_8py_a364b11b1550671bb56014ddf45688488}{font.\+functions.\+yellow} (txt)
\begin{DoxyCompactList}\small\item\em Makes inputed string red. \end{DoxyCompactList}\item 
def \hyperlink{functions_8py_af791023c45a6f4cdb456fb026331fca4}{font.\+functions.\+underline} (txt)
\begin{DoxyCompactList}\small\item\em Makes inputed string underline. \end{DoxyCompactList}\item 
def \hyperlink{functions_8py_a04f1e1329b23c89025a8dc62c930fdb6}{font.\+functions.\+bold} (txt)
\begin{DoxyCompactList}\small\item\em Makes inputed string bold. \end{DoxyCompactList}\item 
def \hyperlink{functions_8py_a74fb9e96ff2f43379b0e3b92266d6513}{font.\+functions.\+italic} (txt)
\begin{DoxyCompactList}\small\item\em Makes inputed string italic. \end{DoxyCompactList}\item 
def \hyperlink{functions_8py_a370bf0fa5e6c47445b1e94b1029efb77}{font.\+functions.\+default} (txt)
\begin{DoxyCompactList}\small\item\em Gives inputed string default formatting. \end{DoxyCompactList}\item 
def \hyperlink{functions_8py_ab9a759f487e4dbefa145af79cff9686b}{font.\+functions.\+highlight} (txt)
\begin{DoxyCompactList}\small\item\em Highlights inputed string. \end{DoxyCompactList}\end{DoxyCompactItemize}


\subsection{Detailed Description}
Defines functions for surronding strings with A\+N\+SI codes for formating text. 



\subsection{Function Documentation}
\index{functions.\+py@{functions.\+py}!blue@{blue}}
\index{blue@{blue}!functions.\+py@{functions.\+py}}
\subsubsection[{\texorpdfstring{blue(txt)}{blue(txt)}}]{\setlength{\rightskip}{0pt plus 5cm}def font.\+functions.\+blue (
\begin{DoxyParamCaption}
\item[{}]{txt}
\end{DoxyParamCaption}
)}\hypertarget{functions_8py_file_ac4016dd2373b79081f110154a3fe2c96}{}\label{functions_8py_file_ac4016dd2373b79081f110154a3fe2c96}


Makes inputed string blue. 

Returns with inputed string with the prefix \textquotesingle{}\textbackslash{}033\mbox{[}34m\textquotesingle{} and suffix \textquotesingle{}\textbackslash{}033\mbox{[}0m\textquotesingle{}. The prefix is the A\+N\+SI code for blue foreground. The suffix is the A\+N\+SI code for reset (white forground, black background).


\begin{DoxyParams}{Parameters}
{\em txt} & A string. Will be displayed as blue. \\
\hline
\end{DoxyParams}
\index{functions.\+py@{functions.\+py}!bold@{bold}}
\index{bold@{bold}!functions.\+py@{functions.\+py}}
\subsubsection[{\texorpdfstring{bold(txt)}{bold(txt)}}]{\setlength{\rightskip}{0pt plus 5cm}def font.\+functions.\+bold (
\begin{DoxyParamCaption}
\item[{}]{txt}
\end{DoxyParamCaption}
)}\hypertarget{functions_8py_file_a04f1e1329b23c89025a8dc62c930fdb6}{}\label{functions_8py_file_a04f1e1329b23c89025a8dc62c930fdb6}


Makes inputed string bold. 

Returns with inputed string with the prefix \textquotesingle{}\textbackslash{}033\mbox{[}1m\textquotesingle{} and suffix \textquotesingle{}\textbackslash{}033\mbox{[}22m\textquotesingle{}. The prefix is the A\+N\+SI code for bold font on. The suffix is the A\+N\+SI code for bold font off.


\begin{DoxyParams}{Parameters}
{\em txt} & A string. Will be made bold. \\
\hline
\end{DoxyParams}
\index{functions.\+py@{functions.\+py}!cyan@{cyan}}
\index{cyan@{cyan}!functions.\+py@{functions.\+py}}
\subsubsection[{\texorpdfstring{cyan(txt)}{cyan(txt)}}]{\setlength{\rightskip}{0pt plus 5cm}def font.\+functions.\+cyan (
\begin{DoxyParamCaption}
\item[{}]{txt}
\end{DoxyParamCaption}
)}\hypertarget{functions_8py_file_a0434b99c6b721734c21210540c8c7dce}{}\label{functions_8py_file_a0434b99c6b721734c21210540c8c7dce}


Makes inputed string cyan. 

Returns with inputed string with the prefix \textquotesingle{}\textbackslash{}033\mbox{[}36m\textquotesingle{} and suffix \textquotesingle{}\textbackslash{}033\mbox{[}0m\textquotesingle{}. The prefix is the A\+N\+SI code for cyan foreground. The suffix is the A\+N\+SI code for reset (white forground, black background).


\begin{DoxyParams}{Parameters}
{\em txt} & A string. Will be displayed as cyan. \\
\hline
\end{DoxyParams}
\index{functions.\+py@{functions.\+py}!default@{default}}
\index{default@{default}!functions.\+py@{functions.\+py}}
\subsubsection[{\texorpdfstring{default(txt)}{default(txt)}}]{\setlength{\rightskip}{0pt plus 5cm}def font.\+functions.\+default (
\begin{DoxyParamCaption}
\item[{}]{txt}
\end{DoxyParamCaption}
)}\hypertarget{functions_8py_file_a370bf0fa5e6c47445b1e94b1029efb77}{}\label{functions_8py_file_a370bf0fa5e6c47445b1e94b1029efb77}


Gives inputed string default formatting. 

Returns with inputed string with the prefix \textquotesingle{}\textbackslash{}033\mbox{[}0m\textquotesingle{}. The prefix is the A\+N\+SI code for reset (white foreground, black background).


\begin{DoxyParams}{Parameters}
{\em txt} & A string. Will be formatted as default. \\
\hline
\end{DoxyParams}
\index{functions.\+py@{functions.\+py}!green@{green}}
\index{green@{green}!functions.\+py@{functions.\+py}}
\subsubsection[{\texorpdfstring{green(txt)}{green(txt)}}]{\setlength{\rightskip}{0pt plus 5cm}def font.\+functions.\+green (
\begin{DoxyParamCaption}
\item[{}]{txt}
\end{DoxyParamCaption}
)}\hypertarget{functions_8py_file_a1a75ea7211b4c6d154d2377b7913f3d2}{}\label{functions_8py_file_a1a75ea7211b4c6d154d2377b7913f3d2}


Makes inputed string green. 

Returns with inputed string with the prefix \textquotesingle{}\textbackslash{}033\mbox{[}32m\textquotesingle{} and suffix \textquotesingle{}\textbackslash{}033\mbox{[}0m\textquotesingle{}. The prefix is the A\+N\+SI code for green foreground. The suffix is the A\+N\+SI code for reset (white forground, black background).


\begin{DoxyParams}{Parameters}
{\em txt} & A string. Will be displayed as green. \\
\hline
\end{DoxyParams}
\index{functions.\+py@{functions.\+py}!highlight@{highlight}}
\index{highlight@{highlight}!functions.\+py@{functions.\+py}}
\subsubsection[{\texorpdfstring{highlight(txt)}{highlight(txt)}}]{\setlength{\rightskip}{0pt plus 5cm}def font.\+functions.\+highlight (
\begin{DoxyParamCaption}
\item[{}]{txt}
\end{DoxyParamCaption}
)}\hypertarget{functions_8py_file_ab9a759f487e4dbefa145af79cff9686b}{}\label{functions_8py_file_ab9a759f487e4dbefa145af79cff9686b}


Highlights inputed string. 

Returns with inputed string with the prefixes \textquotesingle{}\textbackslash{}033\mbox{[}47m\textquotesingle{}, \textquotesingle{}\textbackslash{}033\mbox{[}30m\textquotesingle{}, and\textquotesingle{}\textbackslash{}033\mbox{[}3m\textquotesingle{}, and with the suffix \textquotesingle{}\textbackslash{}033\mbox{[}0m\textquotesingle{}. The prefix is the A\+N\+SI code for italic, white background, and black foreground. The suffix is the A\+N\+SI code for reset (white forground, black background).


\begin{DoxyParams}{Parameters}
{\em txt} & A string. Will be highlighted. \\
\hline
\end{DoxyParams}
\index{functions.\+py@{functions.\+py}!italic@{italic}}
\index{italic@{italic}!functions.\+py@{functions.\+py}}
\subsubsection[{\texorpdfstring{italic(txt)}{italic(txt)}}]{\setlength{\rightskip}{0pt plus 5cm}def font.\+functions.\+italic (
\begin{DoxyParamCaption}
\item[{}]{txt}
\end{DoxyParamCaption}
)}\hypertarget{functions_8py_file_a74fb9e96ff2f43379b0e3b92266d6513}{}\label{functions_8py_file_a74fb9e96ff2f43379b0e3b92266d6513}


Makes inputed string italic. 

Returns with inputed string with the prefix \textquotesingle{}\textbackslash{}033\mbox{[}3m\textquotesingle{} and suffix \textquotesingle{}\textbackslash{}033\mbox{[}23m\textquotesingle{}. The prefix is the A\+N\+SI code for italic on. The suffix is the A\+N\+SI code for italic off.


\begin{DoxyParams}{Parameters}
{\em txt} & A string. Will be displayed as italic. \\
\hline
\end{DoxyParams}
\index{functions.\+py@{functions.\+py}!magenta@{magenta}}
\index{magenta@{magenta}!functions.\+py@{functions.\+py}}
\subsubsection[{\texorpdfstring{magenta(txt)}{magenta(txt)}}]{\setlength{\rightskip}{0pt plus 5cm}def font.\+functions.\+magenta (
\begin{DoxyParamCaption}
\item[{}]{txt}
\end{DoxyParamCaption}
)}\hypertarget{functions_8py_file_abb25d9db89cedb6b2fb0b8371ae196b1}{}\label{functions_8py_file_abb25d9db89cedb6b2fb0b8371ae196b1}


Makes inputed string magenta. 

Returns with inputed string with the prefix \textquotesingle{}\textbackslash{}033\mbox{[}35m\textquotesingle{} and suffix \textquotesingle{}\textbackslash{}033\mbox{[}0m\textquotesingle{}. The prefix is the A\+N\+SI code for magenta foreground. The suffix is the A\+N\+SI code for reset (white forground, black background).


\begin{DoxyParams}{Parameters}
{\em txt} & A string. Will be displayed as magenta. \\
\hline
\end{DoxyParams}
\index{functions.\+py@{functions.\+py}!red@{red}}
\index{red@{red}!functions.\+py@{functions.\+py}}
\subsubsection[{\texorpdfstring{red(txt)}{red(txt)}}]{\setlength{\rightskip}{0pt plus 5cm}def font.\+functions.\+red (
\begin{DoxyParamCaption}
\item[{}]{txt}
\end{DoxyParamCaption}
)}\hypertarget{functions_8py_file_a92cba3e5f82247d09e1e045ce141ec60}{}\label{functions_8py_file_a92cba3e5f82247d09e1e045ce141ec60}


Makes inputed string red. 

Returns with inputed string with the prefix \textquotesingle{}\textbackslash{}033\mbox{[}31m\textquotesingle{} and suffix \textquotesingle{}\textbackslash{}033\mbox{[}0m\textquotesingle{}. The prefix is the A\+N\+SI code for red foreground. The suffix is the A\+N\+SI code for reset (white forground, black background).


\begin{DoxyParams}{Parameters}
{\em txt} & A string. Will be displayed as red. \\
\hline
\end{DoxyParams}
\index{functions.\+py@{functions.\+py}!underline@{underline}}
\index{underline@{underline}!functions.\+py@{functions.\+py}}
\subsubsection[{\texorpdfstring{underline(txt)}{underline(txt)}}]{\setlength{\rightskip}{0pt plus 5cm}def font.\+functions.\+underline (
\begin{DoxyParamCaption}
\item[{}]{txt}
\end{DoxyParamCaption}
)}\hypertarget{functions_8py_file_af791023c45a6f4cdb456fb026331fca4}{}\label{functions_8py_file_af791023c45a6f4cdb456fb026331fca4}


Makes inputed string underline. 

Returns with inputed string with the prefix \textquotesingle{}\textbackslash{}033\mbox{[}4m\textquotesingle{} and suffix \textquotesingle{}\textbackslash{}033\mbox{[}24m\textquotesingle{}. The prefix is the A\+N\+SI code for underline on. The suffix is the A\+N\+SI code for underline off.


\begin{DoxyParams}{Parameters}
{\em txt} & A string. Will be underlined. \\
\hline
\end{DoxyParams}
\index{functions.\+py@{functions.\+py}!yellow@{yellow}}
\index{yellow@{yellow}!functions.\+py@{functions.\+py}}
\subsubsection[{\texorpdfstring{yellow(txt)}{yellow(txt)}}]{\setlength{\rightskip}{0pt plus 5cm}def font.\+functions.\+yellow (
\begin{DoxyParamCaption}
\item[{}]{txt}
\end{DoxyParamCaption}
)}\hypertarget{functions_8py_file_a364b11b1550671bb56014ddf45688488}{}\label{functions_8py_file_a364b11b1550671bb56014ddf45688488}


Makes inputed string red. 

Returns with inputed string with the prefix \textquotesingle{}\textbackslash{}033\mbox{[}33m\textquotesingle{} and suffix \textquotesingle{}\textbackslash{}033\mbox{[}0m\textquotesingle{}. The prefix is the A\+N\+SI code for red foreground. The suffix is the A\+N\+SI code for reset (white forground, black background).


\begin{DoxyParams}{Parameters}
{\em txt} & A string. Will be displayed as red. \\
\hline
\end{DoxyParams}

\hypertarget{Orator_8py}{}\subsection{/\+Terminal\+Talk/server/\+Orator.py File Reference}
\label{Orator_8py}\index{/\+Terminal\+Talk/server/\+Orator.\+py@{/\+Terminal\+Talk/server/\+Orator.\+py}}
\subsubsection*{Classes}
\begin{DoxyCompactItemize}
\item 
class \hyperlink{classserver_1_1Orator_1_1Orator}{server.\+Orator.\+Orator}
\begin{DoxyCompactList}\small\item\em Stores information about a connected client. \end{DoxyCompactList}\end{DoxyCompactItemize}

\hypertarget{pontification_8py}{}\section{/home/bryan/\+Terminal\+Talk/server/pontification.py File Reference}
\label{pontification_8py}\index{/home/bryan/\+Terminal\+Talk/server/pontification.\+py@{/home/bryan/\+Terminal\+Talk/server/pontification.\+py}}
\subsection*{Functions}
\begin{DoxyCompactItemize}
\item 
def \hyperlink{pontification_8py_a63756b0b862790b57cc228bcd4a9b709}{server.\+pontification.\+server\+\_\+pontificate} (verbiage, connections, server, orators)
\begin{DoxyCompactList}\small\item\em Make announcements from server to everyone connected. \end{DoxyCompactList}\item 
def \hyperlink{pontification_8py_a4bf031be83bc6feaeca7bfff6d2d7a8c}{server.\+pontification.\+pontificate} (orator, verbiage, connections, server, client, orators)
\begin{DoxyCompactList}\small\item\em Transmit a clients message to all other connected users. \end{DoxyCompactList}\end{DoxyCompactItemize}


\subsection{Function Documentation}
\index{pontification.\+py@{pontification.\+py}!pontificate@{pontificate}}
\index{pontificate@{pontificate}!pontification.\+py@{pontification.\+py}}
\subsubsection[{\texorpdfstring{pontificate(orator, verbiage, connections, server, client, orators)}{pontificate(orator, verbiage, connections, server, client, orators)}}]{\setlength{\rightskip}{0pt plus 5cm}def server.\+pontification.\+pontificate (
\begin{DoxyParamCaption}
\item[{}]{orator, }
\item[{}]{verbiage, }
\item[{}]{connections, }
\item[{}]{server, }
\item[{}]{client, }
\item[{}]{orators}
\end{DoxyParamCaption}
)}\hypertarget{pontification_8py_file_a4bf031be83bc6feaeca7bfff6d2d7a8c}{}\label{pontification_8py_file_a4bf031be83bc6feaeca7bfff6d2d7a8c}


Transmit a clients message to all other connected users. 


\begin{DoxyParams}{Parameters}
{\em orator} & The sending client\textquotesingle{}s Orator object. \\
\hline
{\em verbiage} & The message displayed. \\
\hline
{\em connections} & A list of all current connections (sockets). \\
\hline
{\em server} & The server\textquotesingle{}s socket. Used to ensure message isn\textquotesingle{}t sent to server itself (this would break pipe). \\
\hline
{\em client} & The sending client\textquotesingle{}s socket. Used to avoid sending message back to client. \\
\hline
{\em orators} & A list of Orator objects for currently connected clients. If a disconnection is discovered while function is being run, it is necessary to remove the object from the list. \\
\hline
\end{DoxyParams}
\index{pontification.\+py@{pontification.\+py}!server\+\_\+pontificate@{server\+\_\+pontificate}}
\index{server\+\_\+pontificate@{server\+\_\+pontificate}!pontification.\+py@{pontification.\+py}}
\subsubsection[{\texorpdfstring{server\+\_\+pontificate(verbiage, connections, server, orators)}{server_pontificate(verbiage, connections, server, orators)}}]{\setlength{\rightskip}{0pt plus 5cm}def server.\+pontification.\+server\+\_\+pontificate (
\begin{DoxyParamCaption}
\item[{}]{verbiage, }
\item[{}]{connections, }
\item[{}]{server, }
\item[{}]{orators}
\end{DoxyParamCaption}
)}\hypertarget{pontification_8py_file_a63756b0b862790b57cc228bcd4a9b709}{}\label{pontification_8py_file_a63756b0b862790b57cc228bcd4a9b709}


Make announcements from server to everyone connected. 

The message will be formatted (see font.\+highlight ) so as to stick out from all other text. Every connected user will see the message.


\begin{DoxyParams}{Parameters}
{\em verbiage} & The message displayed. \\
\hline
{\em connections} & A list of all current connections (sockets). \\
\hline
{\em server} & The server\textquotesingle{}s socket. Used to ensure message isn\textquotesingle{}t sent to server itself (this would break pipe). \\
\hline
{\em orators} & A list of Orator objects for currently connected clients. If a disconnection is discovered while function is being run, it is necessary to remove the object from the list. \\
\hline
\end{DoxyParams}

\hypertarget{TerminalTalk_8py}{}\section{/home/bryan/\+Terminal\+Talk/\+Terminal\+Talk.py File Reference}
\label{TerminalTalk_8py}\index{/home/bryan/\+Terminal\+Talk/\+Terminal\+Talk.\+py@{/home/bryan/\+Terminal\+Talk/\+Terminal\+Talk.\+py}}


Main code for Terminal\+Talk client.  


\subsection*{Variables}
\begin{DoxyCompactItemize}
\item 
string {\bfseries Terminal\+Talk.\+username} = \char`\"{}Anonymous\char`\"{}\hypertarget{TerminalTalk_8py_ae07b468b3c0b52ac1fc41ad0fb0c2e17}{}\label{TerminalTalk_8py_ae07b468b3c0b52ac1fc41ad0fb0c2e17}

\item 
int {\bfseries Terminal\+Talk.\+buffer\+\_\+size} = 2\hypertarget{TerminalTalk_8py_ae58a0cac9c9306b33318a9e11a5518b4}{}\label{TerminalTalk_8py_ae58a0cac9c9306b33318a9e11a5518b4}

\item 
int {\bfseries Terminal\+Talk.\+server\+\_\+port} = 7777\hypertarget{TerminalTalk_8py_a2baa41dd6ac96cb4a2e1ec8c2479dc29}{}\label{TerminalTalk_8py_a2baa41dd6ac96cb4a2e1ec8c2479dc29}

\item 
string {\bfseries Terminal\+Talk.\+server\+\_\+ip} = \char`\"{}192.\+168.\+1.\+77\char`\"{}\hypertarget{TerminalTalk_8py_a09a43c17a8bc338324070b35c7d8e4eb}{}\label{TerminalTalk_8py_a09a43c17a8bc338324070b35c7d8e4eb}

\item 
tuple {\bfseries Terminal\+Talk.\+server\+\_\+address} = ( server\+\_\+ip, server\+\_\+port )\hypertarget{TerminalTalk_8py_aceb3f60c6635e198086fb85c654c36a3}{}\label{TerminalTalk_8py_aceb3f60c6635e198086fb85c654c36a3}

\item 
{\bfseries Terminal\+Talk.\+megaphone} = socket.\+socket(socket.\+A\+F\+\_\+\+I\+N\+ET, socket.\+S\+O\+C\+K\+\_\+\+S\+T\+R\+E\+AM)\hypertarget{TerminalTalk_8py_ad89cd81a2eb78d2872edd23f21bcf2e7}{}\label{TerminalTalk_8py_ad89cd81a2eb78d2872edd23f21bcf2e7}

\item 
{\bfseries Terminal\+Talk.\+request} = megaphone.\+recv(buffer\+\_\+size)\hypertarget{TerminalTalk_8py_a0cf80fdf69e7d0727f17a4b161495403}{}\label{TerminalTalk_8py_a0cf80fdf69e7d0727f17a4b161495403}

\item 
list {\bfseries Terminal\+Talk.\+connections} = \mbox{[}megaphone, sys.\+stdin\mbox{]}\hypertarget{TerminalTalk_8py_a814d1f4829d858a36bd2f40d362a888c}{}\label{TerminalTalk_8py_a814d1f4829d858a36bd2f40d362a888c}

\item 
{\bfseries Terminal\+Talk.\+readables}\hypertarget{TerminalTalk_8py_a67e564d968bd0451bda2cac9e2e6c06a}{}\label{TerminalTalk_8py_a67e564d968bd0451bda2cac9e2e6c06a}

\item 
{\bfseries Terminal\+Talk.\+writables}\hypertarget{TerminalTalk_8py_a3ed15d4c46a7215d5bf4f117993b4b45}{}\label{TerminalTalk_8py_a3ed15d4c46a7215d5bf4f117993b4b45}

\item 
{\bfseries Terminal\+Talk.\+errors}\hypertarget{TerminalTalk_8py_ae98f64b4a6ca8f072bbad015f4deffb7}{}\label{TerminalTalk_8py_ae98f64b4a6ca8f072bbad015f4deffb7}

\item 
{\bfseries Terminal\+Talk.\+missive} = telegraph\+\_\+i.\+recv(buffer\+\_\+size)\hypertarget{TerminalTalk_8py_ac1ebe197a351096e36fc9686fbe3448e}{}\label{TerminalTalk_8py_ac1ebe197a351096e36fc9686fbe3448e}

\item 
{\bfseries Terminal\+Talk.\+verbiage} = sys.\+stdin.\+readline()\hypertarget{TerminalTalk_8py_ac185d4a63a476ebc7c7d7779efe9684c}{}\label{TerminalTalk_8py_ac185d4a63a476ebc7c7d7779efe9684c}

\end{DoxyCompactItemize}


\subsection{Detailed Description}
Main code for Terminal\+Talk client. 

Run this file to use as client. 
\hypertarget{TerminalTalk__server_8py}{}\subsection{/\+Terminal\+Talk/\+Terminal\+Talk\+\_\+server.py File Reference}
\label{TerminalTalk__server_8py}\index{/\+Terminal\+Talk/\+Terminal\+Talk\+\_\+server.\+py@{/\+Terminal\+Talk/\+Terminal\+Talk\+\_\+server.\+py}}


Main code for Terminal\+Talk server.  


\subsubsection*{Variables}
\begin{DoxyCompactItemize}
\item 
int {\bfseries Terminal\+Talk\+\_\+server.\+port} = 7777\hypertarget{TerminalTalk__server_8py_a7624d138e6f12000af1d1d878bce7faa}{}\label{TerminalTalk__server_8py_a7624d138e6f12000af1d1d878bce7faa}

\item 
int {\bfseries Terminal\+Talk\+\_\+server.\+buffer\+\_\+size} = 2\hypertarget{TerminalTalk__server_8py_a02cf935075a72e73b7bb3424e61ab726}{}\label{TerminalTalk__server_8py_a02cf935075a72e73b7bb3424e61ab726}

\item 
tuple {\bfseries Terminal\+Talk\+\_\+server.\+server\+\_\+address} = ( \char`\"{}\char`\"{}, port )\hypertarget{TerminalTalk__server_8py_a87f31d83830565b49acfd0ddc37cb5c6}{}\label{TerminalTalk__server_8py_a87f31d83830565b49acfd0ddc37cb5c6}

\item 
list {\bfseries Terminal\+Talk\+\_\+server.\+connections} = \mbox{[}$\,$\mbox{]}\hypertarget{TerminalTalk__server_8py_af28b68d8c56b2e50f396a608ee0a572b}{}\label{TerminalTalk__server_8py_af28b68d8c56b2e50f396a608ee0a572b}

\item 
list {\bfseries Terminal\+Talk\+\_\+server.\+orators} = \mbox{[}$\,$\mbox{]}\hypertarget{TerminalTalk__server_8py_a308d877fbc3f96a86d035f8262382b3b}{}\label{TerminalTalk__server_8py_a308d877fbc3f96a86d035f8262382b3b}

\item 
{\bfseries Terminal\+Talk\+\_\+server.\+ear\+\_\+trumpet} = socket.\+socket(socket.\+A\+F\+\_\+\+I\+N\+ET, socket.\+S\+O\+C\+K\+\_\+\+S\+T\+R\+E\+AM)\hypertarget{TerminalTalk__server_8py_a8ba5fb6247c4186cc4a60bb9ff5a8b52}{}\label{TerminalTalk__server_8py_a8ba5fb6247c4186cc4a60bb9ff5a8b52}

\item 
{\bfseries Terminal\+Talk\+\_\+server.\+readables}\hypertarget{TerminalTalk__server_8py_aa259fa9187c9aa7bf7b91a13bcad6458}{}\label{TerminalTalk__server_8py_aa259fa9187c9aa7bf7b91a13bcad6458}

\item 
{\bfseries Terminal\+Talk\+\_\+server.\+writables}\hypertarget{TerminalTalk__server_8py_a0dbe0ee9aa6ca9b025a554d2de242c1a}{}\label{TerminalTalk__server_8py_a0dbe0ee9aa6ca9b025a554d2de242c1a}

\item 
{\bfseries Terminal\+Talk\+\_\+server.\+errors}\hypertarget{TerminalTalk__server_8py_ab6b0e558e2673cfc9f9b19860cbc1b63}{}\label{TerminalTalk__server_8py_ab6b0e558e2673cfc9f9b19860cbc1b63}

\item 
{\bfseries Terminal\+Talk\+\_\+server.\+file\+\_\+descriptor}\hypertarget{TerminalTalk__server_8py_a890a4debfe0a4064b0ed6eed57490501}{}\label{TerminalTalk__server_8py_a890a4debfe0a4064b0ed6eed57490501}

\item 
{\bfseries Terminal\+Talk\+\_\+server.\+address}\hypertarget{TerminalTalk__server_8py_a13f81edb6c7fbf3133d5bb27844cb312}{}\label{TerminalTalk__server_8py_a13f81edb6c7fbf3133d5bb27844cb312}

\item 
{\bfseries Terminal\+Talk\+\_\+server.\+moniker} = file\+\_\+descriptor.\+recv(buffer\+\_\+size)\hypertarget{TerminalTalk__server_8py_a839b38bc3b08b2ecfba40fcedbb631fd}{}\label{TerminalTalk__server_8py_a839b38bc3b08b2ecfba40fcedbb631fd}

\item 
string {\bfseries Terminal\+Talk\+\_\+server.\+entrance\+\_\+message} = moniker+\char`\"{} has entered.\char`\"{}\hypertarget{TerminalTalk__server_8py_ac6fd018eabeb6090333333195f53d770}{}\label{TerminalTalk__server_8py_ac6fd018eabeb6090333333195f53d770}

\item 
int {\bfseries Terminal\+Talk\+\_\+server.\+orator\+\_\+index} = 0\hypertarget{TerminalTalk__server_8py_ae899e6c847d41db82934094b87f876a6}{}\label{TerminalTalk__server_8py_ae899e6c847d41db82934094b87f876a6}

\item 
{\bfseries Terminal\+Talk\+\_\+server.\+verbiage} = telegraph\+\_\+i.\+recv(buffer\+\_\+size)\hypertarget{TerminalTalk__server_8py_a4ffcd911df5d4e588d7cef5f88a3915b}{}\label{TerminalTalk__server_8py_a4ffcd911df5d4e588d7cef5f88a3915b}

\item 
string {\bfseries Terminal\+Talk\+\_\+server.\+missive} = orators\mbox{[}orator\+\_\+index\mbox{]}.moniker+\char`\"{} has exited.\char`\"{}\hypertarget{TerminalTalk__server_8py_ae3a0b5651a3b02665b77ef09c94c4e38}{}\label{TerminalTalk__server_8py_ae3a0b5651a3b02665b77ef09c94c4e38}

\end{DoxyCompactItemize}


\subsubsection{Detailed Description}
Main code for Terminal\+Talk server. 

Run this file to start server. 
%--- End generated contents ---

% Index

\newpage
\phantomsection
\clearemptydoublepage
\addcontentsline{toc}{chapter}{Index}
\printindex

\end{document}

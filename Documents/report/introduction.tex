\subsection{Overview}
Test 123
\subsection{The End Product}
In order to determine the application architecture and protocol with which to implement the software it is first necessary to develop a clear picture of what is being built. The following is a description of what an end user can expect when running the \textsl{TerminalTalk} application.

The application will be completely run from the command line (terminal on Linux). There will be no additional graphical user interface beyond what the terminal provides (ANSI characters, cursor animation, colors, etc.). Thus, any interaction the user has with the application will be through their keyboard, in the form of normal input or specialized commands. 

Upon starting the application --- with the option of specifying a moniker (username) --- the user will be connected to a \emph{chat room}. The user's arrival will be announced to all users currently connected:

\begin{displayquote}
 \texttt{[moniker] has entered.}
\end{displayquote}

There will also be available to the user a non-intrusive prompt for the user to deliver a message to the room:

\begin{displayquote}
 \texttt{Talk:}
\end{displayquote}
 
 If the user types a message and then presses \texttt{`Enter'} on their keyboard, the user will simply see their message displayed after the prompt, and then another prompt on a new line. 
 
\begin{displayquote}
 \texttt{Talk: [message inputed by user]} \\
 \texttt{Talk:}
\end{displayquote}

However, any other user in the room will see something like:

\begin{displayquote}
	\texttt{\textbf{\textcolor{cyan}{[moniker1]:}} [message inputed by user1]} \\
	\texttt{Talk:}
\end{displayquote}

Note that the first user's moniker is colored. When a user first begins the application the program randomly assigns a color to the user. However, if a user wishes to change their assigned color, they can use the command \texttt{\textbackslash setcolor}:

\begin{displayquote}
	\texttt{Talk: \textbackslash  setcolor green} \\
	\texttt{\emph{Your color has been set to \textcolor{green}{green}.}}
\end{displayquote}

There is also a command to change the user's moniker:

\begin{displayquote}
	\texttt{Talk: \textbackslash  setmoniker Alan} \\
	\texttt{\emph{Hello Alan.}}
\end{displayquote}

And a command to disconnect:

\begin{displayquote}
	\texttt{Talk: \textbackslash  disconnect} \\
	\texttt{\emph{You have disconnected.}}
\end{displayquote}

Seen from the perspective of other connected users, the output resulting from a user running the above commands might look like this:

\begin{displayquote}
	\texttt{\emph{Alan has entered.}} \\
	\texttt{\textcolor{cyan}{Anonymous:} Hello everyone!} \\
	\texttt{\textcolor{green}{Anonymous:} How's it going?} \\
	\texttt{\textcolor{green}{Alan:} Uh oh, I've got to run!} \\
	\texttt{\emph{Alan has exited.}}
\end{displayquote}

It's expected that the chat room will always be available for a user to connect with at any time. It is also expected that messages are transferred without error. To see the necessity of this, consider if the letter `e` were to be dropped from the word `appeal' in a message. Clearly, such an error could lead to a detrimental misunderstanding.

In a similar vein, it is necessary that messages are received in order. If, for example, a disconnect command were received out of order and before another message, the user would be disconnected before their message got through. This is clearly an unacceptable result. 

